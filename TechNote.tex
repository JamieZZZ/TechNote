\documentclass[11pt,twoside,a4paper]{article}

\usepackage{textcomp}


\begin{document}

\section{Git} % Git Notes
\subsection{Getting Started} % Chapter 1
\begin{enumerate}
  \item \textbf{VCS}: Version Control System; Centralized VCS; Distributed VCS;
  \item \textbf{Git Basics}:
    \begin{itemize}
      \item Snapshots, Not differences
      \item The three states: \textbf{Committed}, .git directory (Repository), data is safely stored in ur local database; \textbf{Modified}, Working Directory, u have changed ur file but have not commited to ur local database yet; \textbf{staged}, Staging Area, u have marked a modified file in its current version to go into ur next commit snapshot.
    \end{itemize}
\end{enumerate}

\subsection{Git Basics} % Chapter 2
\begin{enumerate}
  \item Getting a Git Repository
    \begin{itemize}
      \item Initialize a Repository in an Existing Directory: \textbf{git init}; (then git add file)
      \item Cloning an Existing Repository: \textbf{git clone} [url]
    \end{itemize}
  \item Recording Changes to the Repository
    \begin{itemize}
      \item \textbf{git status}; \textbf{git add}; \textbf{git status -s / --short}, short status; \textbf{git diff}, changed but not staged, --staged / --cached, staged to be committed;
      \item Ignoring files
      \item \textbf{git commit}, -v, add diff content; \textbf{git commit -a -m [msg]}, skip the staging area, add all tracked files and commit.
      \item \textbf{git rm file}, delete a file, git rm -f file, remove it forcely, git rm --cached file, add it in the ignore file list but keep it in the dir.
      \item \textbf{git mv oldname newname}
    \end{itemize}
  \item Viewing the Commit History
    \begin{itemize}
      \item \textbf{git log}, git log -p -[num], including diff info; git log --stat, abbreviated stats; git log --pretty=oneline / short / full / fuller / format:``\%h - \%an, \%ar : \%s''
    \end{itemize}
\end{enumerate}

\subsection{Tips} % Git Tips
\begin{enumerate}
  \item git test
\end{enumerate}
\newpage


\section{Ubuntu} % Ubuntu Notes
\subsection{Tips} % Ubuntu Tips
\begin{enumerate}
  \item \textbf{open pdf via terminal}: evince file.pdf
  \item \textbf{tab commands in gnome-teminal}:
    \begin{itemize}
      \item \textbf{close}: CTRL + SHIRT + W
      \item \textbf{new}: CTRL + SHIRT + T
      \item \textbf{switch}: ALT + [NUMBER]; CTRL + PgDn; CTRL + PgUp;
    \end{itemize}
\end{enumerate}
\newpage


\section{\LaTeX} % LaTeX Note
\subsection{The not so short introduction to \LaTeXe} % The not so short introduction to LaTeXe
\subsubsection{Things You Need to Know} % Chapter 1
\begin{itemize}
  \item \textbf{WYSIWYG}: What you see is what you get.
  \item \TeX{} is a computer program created by Donald E. Knuth. It is aimed at typesetting text and mathematical formulae.
  \item Blank, tab or several consecutive whitespace characters are treated as \textit{one} space. An empty line between two lines of text defnes the end of a paragraph; Several empty lines are treated as one empty line.
  \item \textbf{Sepcial Characters}: \# \$ \% \^{} \{ \} \& \_ \~{} \textbackslash (Note: \textbackslash\textbackslash is for linebreak, use \textbackslash textbackslash instead).
  \item \LaTeX{} ignores whitespace after commands. Put an empty \{ \} to stop \LaTeX from eating up all the whitespaces after commands.
  \item \{ \}: required parameters; [ ] optional parameters
  \item latex file.tex; dvipdf file.dvi
\end{itemize}

\subsubsection{Typesetting Text} % Chapter 2
\begin{itemize}
  \item Paragraph, as the text unit, is the typographical form that should reflect one coherent thought, or one idea. If a new thought begins, a new paragraph should begin with an empty line, instead of line breaker, \textbackslash \textbackslash (start a new line instead of a paragraph).
  \item \textbackslash mbox\{\textit{text}\}: it causes its arguments to be kept together on one line. \textbackslash fbox\{\textit{text}\}: add a visible box in addition.
  \item \textbf{quotation marks}: `` '' ` '
  \item \textbf{dashes and hyphens}:
    \begin{itemize}
      \item \textbf{hyphen}: daughter-in-law
      \item \textbf{en-dash}: pages13--67
      \item \textbf{em-dash}: yes---or no?
      \item \textbf{minus}: $0$, $1$, $-1$
    \end{itemize}
  \item \textbf{Tilde $\sim$}: \$\textbackslash sim\$
  \item \textbf{Slash \slash}: to avoid treating two words as one, use \textbackslash slash instead of /, like read\slash write. but for ratios or units, use / directly, like 5MB/s.
  \item \textbf{Degree Symbol}: 30 \textcelsius{} is 86 \textdegree{}F.
  \item \textbf{Ellipsis}: \textbackslash ldots (low dots), \ldots.
  \item \textbf{Environment}:
    \begin{itemize}
      \item itemize, enumerate, description
      \item flushleft, flushright, center
      \item quote, quotation, verse
      \item abstract
      \item verbatim: print what inside
      \item tabular: table.  \textbackslash \{tabular\}[\textit{pos}]\{table spec\}
    \end{itemize}
\end{itemize}

\subsection{Tips} % LaTeXe Tips
\begin{enumerate}
  \item a
\end{enumerate}
\newpage


\section{Vim} % Vim Note
\subsection{Tips} % Vim Tips
\begin{enumerate}
  \item \textbf{Movement}:
    \begin{itemize}
      \item \textbf{row}: h(left), j(down), k(up), l(right); 3h, 4j, 5k, 6l;
      \item \textbf{move word forward}: w, 3w; \textbf{move word backward}: b, 3b;
      \item \textbf{move to first line}: gg; \textbf{move to last line}: G; \textbf{move to num line}: num + G;
      \item \textbf{move to the end of a line}: \$; SHIRT + A;
      \item \textbf{move to the beginning of a line}: \^{};
    \end{itemize}
  \item Undo: u; Redo: CTRL + R
\end{enumerate}

\end{document}
